\documentclass[a4paper,12pt]{article}
\usepackage[utf8]{inputenc}
\usepackage[T2A]{fontenc}
\usepackage[russian,english]{babel}
\usepackage[pdftex]{graphics}
\DeclareGraphicsExtensions{.pdf,.png,.jpg}
\graphicspath{{pictures/}}
\begin{document}
\begin{center}
Санкт-Петербургский государственный политехнический университет
\\Кафедра компьютерных систем и программных технологий
\end{center}
\vspace*{10em plus .6em minus .5em}

\begin{center}
{\LARGEТелекоммуникационные технологии
\\Лабораторная работа №5
\\Частотная и фазовая модуляция}
\end{center}

\vspace*{5em plus .6em minus .5em}
\begin{flushright}
Выполнил:\\студент гр.33501/4\\Корсков Алексей\\Проверила:\\Богач Н.В.
\end{flushright}

\vspace*{15em plus .6em minus .5em}
\begin{center}
{\smallСанкт-Петербург
\\2018}
\end{center}
\pagestyle{empty}
\newpage
\pagestyle{plain}
{\bfseriesЦель}

Изучение частотной и фазовой модуляции/демодуляции сигнала.

{\bfseriesПостановка задачи}

\begin{itemize}
	\item Сгенерировать однотональный сигнал низкой частоты.
	\item Выполнить фазовую модуляцию/демодуляцию сигнала по закону $u(t)=(U_mcos (\Omega t+ks(t))$, используя встроенную функцию MatLab pmmod, pmdemod
	\item Получить спектр модулированного сигнала.
	\item Выполнить частотную модуляцию/демодуляцию по закону
	
	$u(t)=U_mcos((\omega_0 t+k\int_{0}^{t}s(t)dt+\phi_0)$
	
	используя встроенные функции MatLab fmmod, fmdemod
\end{itemize}

{\bfseriesТеоретическое обоснование}

Частотная модуляция — вид аналоговой модуляции, при котором информационный сигнал управляет частотой несущего колебания. По сравнению с амплитудной модуляцией здесь амплитуда остаётся постоянной. 

ЧМ применяется для высококачественной передачи звукового (низкочастотного) сигнала в радиовещании (в диапазоне УКВ), для звукового сопровождения телевизионных программ, передачи сигналов цветности в телевизионном стандарте SECAM, видеозаписи на магнитную ленту, музыкальных синтезаторах.

Высокое качество кодирования аудиосигнала обусловлено тем, что в радиовещании при ЧМ применяется большая (по сравнению с шириной спектра сигнала АМ) девиация несущего сигнала, а в приёмной аппаратуре используют ограничитель амплитуды радиосигнала для устранения импульсных помех. Такая модуляция называется широкополосной ЧМ. В радиосвязи применяется узкополосная ЧМ с небольшой девиацией частоты несущего сигнала.

Фазовая модуляция - модуляция, при которой фаза несущей изменяется прямо пропорционально информационному сигналу. В случае, когда информационный сигнал является дискретным, то говорят о фазовой манипуляции. По характеристикам фазовая модуляция близка к частотной модуляции. В случае синусоидального модулирующего (информационного) сигнала, результаты частотной и фазовой модуляции совпадают.
\newpage

{\LargeХод работы}
\begin{enumerate}
{\itemСгенерируем однотональный сигнал низкой частоты.
\center{\includegraphics{./pictures/pic1.png} \\ Рис.1 Сигнал}
\\}

{\itemВыполним фазовую модуляцию, используя функцию pmmod
\center{\includegraphics{./pictures/pic2.png} \\ Рис.2 Амплитудная модуляция}
\center{\includegraphics{./pictures/pic3.png} \\ Рис.3 Спектр сигнала}
\\}

{\itemВыполним демодуляцию ФМ-сигнала.
\center{\includegraphics{./pictures/pic4.png} \\ Рис.4 Модуляция с подавлением несущей}
\center{\includegraphics{./pictures/pic5.png} \\ Рис.5 Спектр сигнала}
\\}

{\itemВыполним частотную модуляцию, используя функцию fmmod.
\center{\includegraphics{./pictures/pic6.png} \\ Рис.6 Однополосная модуляция}
\center{\includegraphics{./pictures/pic7.png} \\ Рис.7 Спектр сигнала}
\\}

{\itemВыполним демодуляцию ЧМ-сигнала.
\center{\includegraphics{./pictures/pic8.png} \\ Рис.8 Сигнал после синхронного детектирования}
\center{\includegraphics{./pictures/pic9.png} \\ Рис.9 Спектр сигнала}
\\}

{\bfseries\LARGEВывод}

В ходе выполнения лабораторной работы исследована фазовая и частотная модуляция/демодуляция сигналов. Модуляция сигналов находит широкое применение в телекоммуникационных технологиях. Например, используется для высококачественной передачи звукового сигнала в теле- и радиовещании, в сотовой телефонной связи и других системах.

\end{enumerate}
\end{document}
