\documentclass[a4paper,12pt]{article}
\usepackage[utf8]{inputenc}
\usepackage[T2A]{fontenc}
\usepackage[russian,english]{babel}
\usepackage[pdftex]{graphics}
\DeclareGraphicsExtensions{.pdf,.png,.jpg}
\graphicspath{{pictures/}}
\begin{document}
\begin{center}
Санкт-Петербургский государственный политехнический университет
\\Кафедра компьютерных систем и программных технологий
\end{center}
\vspace*{10em plus .6em minus .5em}

\begin{center}
{\LARGEТелекоммуникационные технологии
\\Лабораторная работа №7
\\Помехоустойчивое кодирование}
\end{center}

\vspace*{5em plus .6em minus .5em}
\begin{flushright}
Выполнил:\\студент гр.33501/4\\Корсков Алексей\\Проверила:\\Богач Н.В.
\end{flushright}

\vspace*{15em plus .6em minus .5em}
\begin{center}
{\smallСанкт-Петербург
\\2018}
\end{center}
\pagestyle{empty}
\newpage
\pagestyle{plain}
{\bfseriesЦель}

Изучение методов помехоустойчивого кодирования и сравнение их свойств

{\bfseriesПостановка задачи}

\begin{itemize}
\item Провести кодирование/декодирование сигнала, полученного с помощью функции randerr кодом Хэмминга 2-мя способами: с помощью встроенных функций encode/decode, а также через создание проверочной и генераторной матриц и вычисление синдрома. Оценить корректирующую способность кода.
\item Выполнить кодирование/декодирование циклическим кодом, кодом БЧХ, кодом Рида-Соломона. Оценить корректирующую способность кода.
\end{itemize}

{\bfseriesТеоретическое обоснование}

Кодированием и декодированием (в широком смысле) называют любое преобразование сообщения в сигнал и обратно, сигнала в сообщение, путем установления взаимного соответствия. Преобразование следует считать оптимальным, если в конечном итоге производительность источника и пропускная способность канала окажутся равными, т.е. возможности канала будут полностью использованы. Данное преобразование разбивается на два этапа:
\begin{itemize}
\item модуляция-демодуляция, позволяющая осуществить переход от непрерывного сигнала радиоканала к дискретному;
\item кодирование-декодирование (в узком смысле), во время которого все операции выполняются над последовательностью символов.
\end{itemize}

В свою очередь, кодирование-декодирование делится на два противоположных по своим действиям действиям этапа:
\begin{itemize}
\item устранение избыточности в принимаемом от источника сигнале (экономное кодирование);
\item внесение избыточности в передаваемый по каналу цифровой сигнал (помехоустойчивое или избыточное кодирование) для повышения достоверности передаваемой информации.
\end{itemize}

Циклический код — подкласс линейных кодов, обладающие следующим свойством: циклическая подстановка символов в кодированном блоке дает другое возможное кодовое слово того же кода.

К циклическим кодам относятся коды Хэмминга, которые являются одним из немногочисленных примеров совершенных кодов. Длина кодированного блока равна 2m-1. Порождающая и проверочная матрицы для кодов Хэмминга генерируются функцией hammgen. 

Среди циклических кодов широкое применение нашли коды Боуза-Чоудхури-Хоквингема (БЧХ). Для работы с ними есть функции bchenco (кодирование) и bcddeco (декодирование). Функция bchpoly позволяет расчитывать и считывать параметры или порождающий полином для двоичных кодов БЧХ.

Частным случаем БЧХ кодов являются коды Рида-Соломона - подкласс циклических блочных кодов. Это единственные поддерживаемые пакетом Communications недвоичные коды. Для работы с этим кодом есть функции rsenco (кодирование) и rsdeco (декодирование). Функции rsencof и rsdencof осуществляют кодирование и декодирование текстового файла. Функция rspoly генерирует порождающие полиномы для кодов Рида-Соломона.
\newpage

{\LargeХод работы}
\begin{enumerate}
{\itemРезультат кодирования/декодирования кодом Хэмминга с использованием стандартных функций:
\center{\includegraphics{./pictures/pic1.png} \\ Рис.1 Результат работы программы}
\\}

{\itemРезультат кодирования/декодирования кодом Хэмминга с использованием проверочной и генераторной матриц и вычисление синдрома:
\center{\includegraphics{./pictures/pic2.png} \\ Рис.2 Результат работы программы}
\\}

{\itemРезультат кодирования/декодирования циклическим кодом:
\center{\includegraphics{./pictures/pic3.png} \\ Рис.3  Результат работы программы}
\\}

{\itemРезультат кодирования/декодирования кодом БЧХ:
\center{\includegraphics{./pictures/pic4.png} \\ Рис.4 Результат работы программы}
\\}

{\itemРезультат кодирования/декодирования кодом Рида-Соломона:
\center{\includegraphics{./pictures/pic5.png} \\ Рис.5 Результат работы программы}
\\}

{\bfseries\LARGEВывод}

В ходе данной работы были получены навыки кодирования цифровых сигналов. Кодирование таких сигналов происходит по принципу избыточности. Каждый из исследованных кодов имеет свои преимущества и недостатки, поэтому использование конкретного из них должно быть обусловлено постановкой определенной задачи. Код Хэмминга достаточно простой в использовании, не требует больших мощностей. Однако он может исправить только одну допущенную ошибку в переданном сообщении. Код Рида-Соломона способен исправлять несколько ошибок, так же он может оперировать десятичными числами, а не только двоичными.
\end{enumerate}
\end{document}
